\section{Introduction}
With the explosive technological development of the Graphics Processor Unit (GPU) hardware nowdays,
Computer Graphics (CG) generation software is given the opportunity to implement increasingly complex image
creation algorithms that simulate the real world. CG software is used in various areas, from the movie
industry to create virtual worlds and special effects, to interactive applications in personal entertainment
industry (Games). The CG algorithms used vary mostly from the amount of interaction needed, as with the more
time consuming algorithms (Raytracing family) it is not possible to achieve satisfying interactive image
generation rates. For these algorithms to achieve high image generation rates (framerates) some
simplifications\footnote{See for example, `Direct --- Indirect lighting separation'} and optimizations\footnote{See
for example, `Deferred Rendering'} must be done. The simplifications mostly take away the small visual phenomena
that are hard to calculate in realtime like recursive reflections, light refraction and accurate indirect lighting
contribution to an object. The optimizations usually are about respecting how the hardware operates and allowing
it to do it faster, eliminating redundant calculations or batching similar ones. Furthermore, when implementing CG
software a good architecture is necessary to be established in order to manage the various algorithms and inputs
that contribute to the final result. Such inputs are geometry data from various 3D model file formats, color data
from various image formats and custom configuration files that can help in the external\footnote{Out of source}
modification of the software. We explore all these aspects that where taken into account when developing this
project, along with the gritty details of their implementations.
