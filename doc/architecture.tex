\section{Architecture}
Before moving on to the various algorithms used to compose the engine's visual output, a few
words need to be said about its architecture.

At its core, the engine operates on a single threaded loop with constant update time and
variable rendering. The loop ends when it gets an exit signal. Each set of render frames,
executed between updates, are interpolated for the purpose of achieving more fluent interactions
on higher end hardware. From an extremely distant point of view, that is all there is to it.
Naturally there are various utilities that help with those two tasks and they will be briefly
explained in the following sections.

\subsection{Window}
Another decoupled yet important part of the engine is the Window. All the window creating logic
is separated from everything else, hence it is easy to implement window utilities (e.g.\ change
title, subscribe to events) or even replace the whole context creation backend if needed.

\subsection{State Management}
Each scene, is considered a different state\footnote{A scene can be anything that appears on
screen. It can be a main menu screen or credits or a game scene etc\dots}The state management
system allows selecting the next state of each screen and defining each scene's resources
so they can be loaded independently. When the screen changes, a loading screen appears which loads
all assets the scene is going to need.

\subsection{Renderers}
The result of rendering is the outcome of the combined efforts of quite a few Renderers, each
with his own purpose. Namely, so far there are Renderers for:

\begin{itemize}
\item Management of defferred rendering and all the other Renderers. This is the main Renderer.
\item AABBs
\item Debug view\footnote{Debug view is a mode where the visual output includes separate steps
      of the deferred rendering. For example a view with only the normals or a view with the
      shadow maps}
\item Runtime shell
\item Shadow Maps
\item Skysphere and Skybox
\item Text
\end{itemize}

\subsection{Assets and Resources}
One of the governing aspects of a game engine is the Assets loading and management. For the
purpose of promoting modularization and eliminating code duplication a convention was made.

\begin{itemize}
\item Assets that are loaded (from disk, network, etc..) and stored to RAM
\item Resources that are loaded to the GPU\@.
\end{itemize}

\noindent For example a model asset has the following lifecycle:

\begin{enumerate}
\item Its model file will be loaded from disk to memory buffer
\item It will idle for sometime
\item It will be added to Model Store which will uploaded it to GPU (its vertices, texture coords,
      etc..)
\end{enumerate}

Basically, assets are discriminated based on their state. When one starts thinking of them that
way, it is easy to notice that many modules do not actually require the asset itself. For
instance, the Rendering module only expects an identifier for the loaded asset; it has no use
for the actual loaded data. Using that convention has proven a wise decision since it speeded
up debugging and developing.

Throughout the code there are many independent asset loaders, for images, models, scenefiles etc..
and \textit{Asset Stores}, which get as input the loaded data, upload it to the GPU and save an
identifier to access them. A store's assets can be referenced using a string key.

\noindent For the time being, the supported asset types are the following:

\begin{itemize}
\item Geometry
\item Image
\item Material
\item Scene
\item Font
\end{itemize}

\subsection{Shader}
One of the greatest and most annoying problems a graphics developer encounters is the hassle of
having to restart the whole program for just a simple alteration to a shader. To address that
problem the following convention was made:

\noindent Shaders are divided to two categories:

\begin{itemize}
\item Static shaders, embedded in source code
\item Normal shaders, in their own glsl files
\end{itemize}

\noindent Static shaders are needed for independent features that are not likely to change often,
just like the shaders for Shadow Maps or Cubemaps. Normal shaders, on the other hand, are those
that contain all lighting equations (among other thigns) and that usually means that they are a
lot more likely to change.\\
There were two features developed for that reason.

\subsubsection{Shader Modularization}
As the engine grows larger the features usually get more complex. GLSL is not an exception to that
rule. The engine's GLSL is enchanced with a preprocessing mechanism that promotes code reusability
and modularity. For the time being, the only preprocessor directives implemented are those of
\textit{\#include} and \textit{\#module} which are respectively used to include a module to the
current one and declare a file as a separate module. Note that multiple includes are ignored by
default\footnote{Just like using include guards in C/C++}.

\subsubsection{Shader Reloading}
As stated earlier, many shaders needs to be reloaded many times during development. Restarting
the program meant reloading all the assets which means waiting time. Shader Reloading feature
comes to the rescue as it eliminates the need for program restarting for mere changes in shaders.
They can be reloaded on the fly, and if there is an error, a message box appears without stopping
the program.

\subsection{Embedded Shell}
An embedded shell is implemented in order to provide a way to give commands on runtime.
