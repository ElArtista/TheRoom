\section{Lighting}
\subsection{Preface}
One of the most important factors that make a scene realistic is how the light interacts with the objects in it.
Through the time rendering techniques progressed from various empirical models (e.g. Phong models), that required
tweaking from the artists in order to make results seem more real under different lighting setups, to physical based
algorithms (e.g. Microfacet models) that calculate the correct object lighting given the lighting environment and
its material properties. The Physical Based models are based around the laws of physics on how light interacts with
matter depending on a number of environmental variables.

\subsection{Background}
All the physical based rendering models try to solve a simplified version of the rendering equation~\cite{lighting:ref32}.
This equation is a function which gives the outgoing light in a particular direction $\omega_o$ from a point $p$ on a surface
and looks like this:

\begin{equation}
\label{eq:rendeq}
L_o(p,\omega_o) = L_e(p, \omega_o) + \int_\Omega f_r(p,\omega_i,\omega_o)L_i(p,\omega_i)n\cdot\omega_i \ d\omega_i
\end{equation}

Breaking down the equation we get:

\begin{equation}
\underbrace{L_o(p,\omega_o)}_\text{Outgoing Light} = \underbrace{L_e(p, \omega_o)}_\text{Emitted Light} + \underbrace{\int_\Omega f_r(p,\omega_i,\omega_o)L_i(p,\omega_i)n\cdot\omega_i \ d\omega_i}_\text{Integral}
\end{equation}

According to the equation the \textit{outgoing light} in a particular direction from a point on a surface is the sum of the
sum of the \textit{emitted light} from that point and the sum of all the incoming light from all directions in a hemisphere
above the point p (the orientation of the hemisphere is determined by the normal n).

Breaking down the integral we get:
\begin{equation}
\int_\Omega \underbrace{f_r(p,\omega_i,\omega_o)}_\text{BRDF}\underbrace{L_i(p,\omega_i)}_\text{Incoming Light}\underbrace{n\cdot\omega_i}_\text{Normal Attenuation}\ d\omega_i
\end{equation}

Where BRDF is a function returning the ratio of the amount of light reflected in a particular direction $\omega_o$, to the
amount received from another direction $\omega_i$ (more details later), incoming light is the light arriving at the point p
from the direction $\omega_i$, and the normal attenuation is an attenuation factor that makes the incoming light at $p$
dependent on the cosine of the angle between the normal $n$ and the incoming light direction $\omega_i$. Note that the
incoming light doesn't have to come from a light source (direct light), it may have been reflected or refracted from another
point int the scene (indirect light).

\subsection{Diffuse and Specular lighting}
When light hits a surface there are three possible outcomes. Light may be \textbf{absorbed} by the material, light may be
\textbf{transmitted} through to the other side or light may be \textbf{reflected} back.
Reflected light can be divided into two sub-types, specular reflection and diffuse reflection. Specular reflection reflects
all light near the same angle, whereas diffuse reflection reflects in a broad range of directions.
The Cook-Torrance model that is also used in the current implementation also makes this separation by describing the BRDF
function as:

$$f_r = k_d f_{lambert} + k_s f_{cook-torrance}$$

Where $f_{lambert}$ describes the diffuse component, $f_{cook-torrance}$ describes the specular component, $k_d$ is the amount
of incoming radiance that gets diffused and $k_s$ is the amount of light that is specularly reflected. If a given material
exhibits a strong diffusive behaviour it have a high value for $k_d$, while if it behaves more like a mirror it will have high $k_s$.
For the diffuse BRDF has been choosen the Lambert's BRDF that is equal to:

$$f_{lambert} = \frac{c}{\pi}$$

where c is the surface colour. The specular Cook-Torrance BRDF is defined as:

$$f_{cook-torrance} = \frac{DFG}{(4 \omega_o \cdot n)(\omega_i \cdot n)}$$

Where D is the distribution function, F is the fresnel function and G is the geometry function.
These terms exist, as the Cook-Torrance model falls into the category of the \textbf{Microfacet Models} which revolves around the idea
that rough surfaces can be modelled as a collection of small microfacets, each of them is assumed to be a very small perfect mirror,
and their distribution and orientation define how the light is reflected at a large scale.
